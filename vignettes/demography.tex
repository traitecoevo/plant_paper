\documentclass[10pt,twoside]{article}

\usepackage{suppmat}
\usepackage{listings}

\newcommand{\plant}{\textsc{plant}}

\usepackage{graphicx}
% We will generate all images so they have a width \maxwidth. This means
% that they will get their normal width if they fit onto the page, but
% are scaled down if they would overflow the margins.
\makeatletter
\def\maxwidth{\ifdim\Gin@nat@width>\linewidth\linewidth\else\Gin@nat@width\fi}
\def\maxheight{\ifdim\Gin@nat@height>\textheight\textheight\else\Gin@nat@height\fi}
\makeatother
\setkeys{Gin}{width=\maxwidth,height=\maxheight,keepaspectratio}

\title{plant: A package for modelling forest trait ecology \& evolution:
\emph{Modelling demography of plants, patches and metapopulations}}
\date{}

\usepackage[sort&compress]{natbib}
\bibliographystyle{../mee}

\begin{document}

\maketitle


\section{Introduction}\label{introduction}
This vignette outlines methods used to model demography in the {\plant}
package, using methods from \citet{Deroos-1997}; \citet{Kohyama-1993};
\citet{Moorcroft-2001}, \citet{Falster-2011}; and \citet{Falster-2015}.
We first outline the system dynamics and then describe the numerical
techniques used to solve the equations. Variable definitions and units are
summarised in Table \ref{tab:definitions}.

\section{System dynamics}\label{system-dynamics}

\subsection{Individual plants}\label{individual-plants}

Consider the dynamics of an individual plant. Throughout we refer to a
plant as having traits \(x\) and size \(h\), notionally it's height. The
plant grows in an environment \(E\), a function giving the distribution
of light with respect to height. Ultimately \(E\) depends on the
composition of the patch at age \(a\). To indicate this dependence we
write \(E_a\). Now let the functions \(g(x,h,E_a)\), \(d(x,h,E_a)\) and
\(f(x,h,E_a)\) denote the growth, death, and fecundity rates of the
plant. Then,
\begin{equation} \label{eq:size}
  h(x, a_0, a) = h_0 + \int_{a_0}^{a} g(x, h(x, a_0, a^\prime),E_{a^\prime}) \, \rm{d}a^\prime
\end{equation}
is the trajectory of plant height,
\begin{equation} \label{eq:survivalIndividual}
  S_{\rm I} (x, a_0, a) = S_{\rm G} (x,h_0, E_{a_0}) \, \exp\left(- \int_{a_0}^{a} d(x,h(x, a_0, a^\prime), E_{a^\prime}) \, {\rm d} a^\prime \right)
\end{equation}
is the probability of survival \(S_{\rm I}\) within the patch, and
\begin{equation} \label{eq:tildeR1}
  \tilde{R}(x, a_0, a) =\int_{a_0}^{a} f(x, h(x, a_0, a^\prime), E_{a^\prime}) \, S_{\rm I} (x, a_0, a^\prime) \, {\rm d} a^\prime
\end{equation}
is the cumulative seed output for the plant from its birth at age
\(a=a_0 \rightarrow a\), and where the term \(S_{\rm G} (x,h_0, E_{a_0})\)
denotes survival through germination.

The notational complexity required in Eqs. \ref{eq:size} -
\ref{eq:tildeR1} potentially obscures an important point: Eqs.
\ref{eq:size} - \ref{eq:tildeR1} are general, non-linear solutions to
integrating growth, mortality and fecundity functions over time.

\subsection{Patches of competing plants / Size-structured
populations}\label{patches-of-competing-plants-size-structured-populations}

Let us now consider a patch of competing plants. At any age \(a\), the
patch is described by the density-distribution \(n(x,h,a)\) of plants
with traits \(x\) and height \(h\). In a finite-sized patch, \(n\) is a
collection of delta-peaks, whereas as in a very (infinitely) large patch
\(n\) is a continuous distribution. In either case, the demographic
behaviour of the plants within the patch is given by Eqs. \ref{eq:size}
- \ref{eq:tildeR1}. Integrating the dynamics over time is complicated by
two other factors: i) plants interact, thereby altering \(E_a\) with
age; and (ii) new individuals may establish, expanding the the system of
equations.

In the current version of {\plant} plants interact by shading one another.
Following standards biophysical principles, we let canopy openness
\(E_a(z)\) at height \(z\) in a patch of age \(a\) decline exponentially
with the total amount of leaf area above \(z\), i.e.
\begin{equation} \label{eq:light}
  E_a(z) = \exp \left(-c_{ext}  \sum_{i=1}^{N} \int_{0}^{\infty} a_l(h) \, Q(z, h) \, n(x_i,h,a) \, {\rm d}h \right),
\end{equation}
where \(a_l(h)\) is total leaf area and \(Q(z, h)\) is fraction of
this leaf area held above height \(z\) for plants size \(h\),
\(c_{ext}\) is the light extinction coefficient, and \(N\) is the number
of species.

Assuming patches are large, the dynamics of \(n\) can be modelled
deterministically via the following Partial Differential Equation (PDE)
\citep{Kohyama-1993, Deroos-1997, Moorcroft-2001}:
\begin{equation} \label{eq:PDE} 
  \frac{\partial }{\partial a} n(x,h,a)= -d(x,h, E_a) \, n(x,h,a)-\frac{\partial }{\partial h} \left[g(x,h,E_a) \, n(x,h,a)\right].
\end{equation}
(See section \ref{derivation-of-pde-describing-size-structured-dynamics} for derivation.)

Eq. \(\ref{eq:PDE}\) has two boundary conditions. The first links the
flux of individuals across the lower bound \((h_0)\) of the size
distribution to the rate at which seeds arrive in the patch, \(y_x\):
\begin{equation} \label{eq:BC1}
  n(x,h_0,a_0)  = \left\{
  \begin{array}{ll}   \frac{y_x  \, S_{\rm G} (x, h_0, E_{a_0}) }{ g(x,h_0, E_{a_0}) }  & \textrm{if } g(x,h_0, E_{a_0}) > 0 \\
  0 & \textrm{otherwise.}
  \end{array} \right.
\end{equation}
The function \(S_{\rm G} (x, h_0, E_{a_0})\) denotes survival through
germination and must be chosen such that
\(S_{\rm G} (x, h_0, E_{a_0}) / g(x,h_0, E_{a_0}) \rightarrow 0\) as
\(g(x,h_0, E_{a_0}) \rightarrow 0\) to ensure a smooth decline in
initial density as conditions deteriorate \citep{Falster-2011}.

The second boundary condition of Eq. \(\ref{eq:PDE}\) gives the size
distribution for patches when \(a=0\). Throughout we consider only
situations where we start with an empty patch, i.e.
\begin{equation} \label{eq:BC2} n\left(x,h,0\right) =0,
\end{equation}
although non--zero distributions could be specified
\citep[e.g][]{Moorcroft-2001}.

\subsection{Age-structured distribution of
patches}\label{age-structured-distribution-of-patches}

Let us now consider the abundance of patches age \(a\) in the landscape.
Let \(a\) be time since last disturbance, \(p(a)\) be the
frequency-density of patches age \(a\), and \(\gamma(a)\) be the
age-dependent probability that a patch of age \(a\) is transformed into
a patch of age 0 by a disturbance event. Here we focus on the situation
where the age structure has reached an equilibrium state, which causes
the PDE to reduce to an ordinary differential equation (ODE) with
respect to patch age. (See section
\ref{derivation-of-pde-describing-age-structured-dynamics} for derivation and
non-equilibrium case). The dynamics of \(p\) are given by
\citep{Vonfoerster-1959, Mckendrick-1926}:
\begin{equation} \label{eq:agepde}
\frac{{\rm d}}{{\rm d} a} p(a)  = -\gamma(a) \, p(a) ,
\end{equation}
with boundary condition
\begin{equation}  p(0)  = \int_0^\infty \gamma(a) \, p(a) \, {\rm d} a.
\end{equation}
The probability a patch remains undisturbed from \(a_0\) to \(a\) is
then
\begin{equation} \label{eq:survivalPatch}
  S_{\rm P} (a_0,a) = \exp\left(-\int_{a_0}^{a} \gamma(a^\prime) \, {\rm d} a^\prime \right).
\end{equation}

The above equations lead to an equilibrium distribution of patch-ages
\begin{equation} p(a) = p(0) S_{\rm P} (0,a),
\end{equation}
where
\begin{equation}
  p(0) = \frac1{\int_0^\infty S_{\rm P} (0,a) {\rm d}a},
\end{equation}
is the average lifespan of a patch and the frequency-density of patches
age \(0\).

The default approach in {\plant} is to assume  $\gamma(a)$ is an
increasing function of patch age, which leads to a Weibull distribution
(see section \ref{derivation-of-pde-describing-age-structured-dynamics}).

\subsection{Trait-, size- and patch-structured
metapopulations}\label{trait--size--and-patch-structured-metapopulations}

Consider a large area of habitat where: i) disturbances (such as fires,
storms, landslides, floods, or disease outbreaks) strike patches of the
habitat on a stochastic basis, killing individuals within affected
patches; ii) individuals compete for resources within patches, but the
spatial scale of competitive interaction means interactions among
individuals in adjacent patches are negligible; and iii) there is high
connectivity via dispersal between all patches in the habitat, allowing
empty patches to be quickly re-colonised. Such a system can be modelled
as a metapopulation (sometimes called metacommunity for multiple
species). The dynamics of this metapopulation are described by PDES in
Eqs. \ref{eq:agepde} and \ref{eq:PDE}.

The seed rain of each species in the metapopulation is given by rate at
which seeds are produced across all patches,
\begin{equation}  \label{eq:seed_rain} 
  y_x = \int_0^{\infty} p(a)  \int_0^{\infty}  S_D \, f(x,h,E_a) \, n(x,h,a)\,{\rm d} m \, {\rm d} a,
\end{equation}
where \(S_D\) is the average survival of seeds during dispersal.

A convenient feature of Eqs. \(\ref{eq:PDE}\) - \(\ref{eq:BC2}\) is that
the dynamics of a single patch scale up to give the dynamics of the
entire metapopulation. Note that the rate offspring arrive from the
disperser pool, \(y_x\), is constant for a metapopulation at
equilibrium. Combined with the assumption that all patches have the same
initial (empty) size distribution, the assumption of constant seed rain
ensures all patches show the same temporal behaviour, the only
difference between them being the ages at which they are disturbed.

To model the temporal dynamics of an archetypal patch, we need only a
value for \(y_x\). The numerical challenge is therefore to find the
right value for \(y_x\), by solving in Eqs. \ref{eq:BC1} and
\ref{eq:seed_rain} as simultaneous equations.

\subsection{Emergent properties of
metapopulation}\label{emergent-properties-of-metapopulation}

Summary statistics of the metapopulation are obtained by integrating
over the density distribution, weighting by patch abundance \(p(a)\).
The total number of individuals in the metapopulation is given by
\begin{equation}
  \hat{n}(x) = \int_{0}^{\infty} \int_{0}^{\infty}p(a) \, n(x,h,a) \, {\rm d}a \, {\rm d}h;
\end{equation}
and the average density of plants size \(h\) by
\begin{equation}
  \bar{n}(x,h) = \int_{0}^{\infty}p(a) \, n(x,h,a) \, {\rm d}a.
\end{equation}

Average values for other quantities can also be calculated. Let
\(\phi(x, h, E_a)\) be a demographic quantity of interest, such as
growth rate, mortality rate, light environment, or size. The average
value of \(\phi\) across the metapopulation is
\begin{equation}
  \hat{\phi}(x) = \frac1{\hat{n}(x) }\int_{0}^{\infty} \int_{0}^{\infty} p(a) \, n(x,h,a) \, \phi(x,h,E_a) \, {\rm d}a \, {\rm d}h,
\end{equation}
while the average value of \(\phi\) for plants size \(h\) is
\begin{equation}\bar{\phi}(x,h) = \frac1{\bar{n}(x,h) }\int_{0}^{\infty}p(a)  \, n(x,h,a) \, \phi(x,h,E_a)\, {\rm d}a.
\end{equation}

When calculating average mortality rate, one must decide whether
mortality due patch disturbance is included. Non-disturbance mortality
is obtained by setting \(\phi(x,h, E_a) = d(x,h,E_a)\), while total
mortality due to growth processes and disturbance is obtained by setting
\(\phi(x,h, E_a) = d(x,h,E_a)+ \gamma(a) S_{\rm P}(0,a).\)

\subsection{Invasion fitness}\label{invasion-fitness}

Let us now consider how we can estimate the fitness of a rare individual
with traits \(x^\prime\) growing in the environment of a resident
community with traits \(x\). We will focus on the phenotypic components
of fitness -- i.e.~the consequences of a given set of traits for growth,
fecundity and mortality -- taking into account the non-linear effects of
competition on individual success, but ignoring the underlying genetic
basis for the trait determination. We also adhere to the standard
conventions in such analyses in assuming that the mutant is sufficiently
rare to have a negligible effect on the environment where it is growing.

Invasion fitness is most correctly defined as the long-term per capita
growth rate of a rare mutant population growing in the environment
determined by the resident strategy \citep{Metz-1992}. Calculating
per-capita growth rates, however, is particularly challenging in a
structured metapopulation model \citep{Gyllenberg-2001, Metz-2001}. As
an alternative measure of fitness, we can use the basic reproduction
ratio, which gives the expected number of new dispersers arising from a
single dispersal event. Evolutionary inferences made using the basic
reproduction ratio will be similar to those made using per-capita growth
rates for metapopulations at demographic equilibrium
\citep{Gyllenberg-2001, Metz-2001}.

Let \(R\left(x^\prime,x\right)\) be the basic reproduction ratio of
individuals with traits \(x^\prime\) growing in the competitive
environment of the resident traits \(x\). Recalling that patches of age
\(a\) have density \(p(a)\) in the landscape, it follows that any seed
of \(x^\prime\) has probability \(p(a)\) of landing in a patch age
\(a\). The basic reproduction ratio for individuals with traits
\(x^\prime\) is then:
\begin{equation} \label{eq:InvFit}
  R\left(x^\prime,x\right)=\int _0^{\infty }p\left(a\right) \, \tilde{R}\left(x^\prime,a, \infty \right)\, {\rm d}a ,
\end{equation}
where \(\tilde{R}\left(x^\prime,a_0,a \right)\) is the expected number
of dispersing offspring produced by a single dispersing seed arriving in
a patch of age \(a_0\) up until age \(a\)
\citep{Gyllenberg-2001, Metz-2001}.
\(\tilde{R}\left(x^\prime,a,\infty\right)\) is calculated by integrating
an individual's fecundity over the expected lifetime the patch, taking
into account competitive shading from residents with traits \(x\), the
individual's probability of surviving, and its traits via the equation:
\begin{equation} \label{eq:tildeR}
  \tilde{R}(x^\prime, a_0, a) =\int_{a_0}^{a}  S_D \, f(x^\prime, h(x^\prime, a_0, a^\prime), E_{a^\prime}) \, S_{\rm I} (x^\prime, a_0, a^\prime) \, S_{\rm P} (a_0,a^\prime) \,  {\rm d} a^\prime.
\end{equation}

\section{Approximating system dynamics using the escalator boxcar train
(EBT)}\label{approximating-system-dynamics-using-the-escalator-boxcar-train-ebt}

Our approach for solving the trait-, size- and patch-structured
population dynamics described by Eqs. \ref{eq:size} - \ref{eq:tildeR} is
based on the Escalator Boxcar Train technique (EBT)
\citep{Deroos-1997, Deroos-1992, Deroos-1988}. The EBT solves the PDE
describing development of \(n(x,h,a)\) (Eq. \(\ref{eq:PDE}\)) by
approximating the density function with a collection of cohorts spanning
the size spectrum. Following a disturbance, a series of cohorts are
introduced into each patch. These cohorts are then transported along the
characteristics of Eq. \(\ref{eq:PDE}\) according to the
individual-level growth function. Characteristics are curves along which
Eq. \(\ref{eq:PDE}\) becomes an ordinary differential equation (ODE);
biologically, these are the growth trajectories of individuals, provided
they do not die.

The original EBT \citep{Deroos-1997, Deroos-1992, Deroos-1988} proceeds
by approximating the first and second moments of the density
distribution \(n \left( x,h,a \right)\) within a series of ``cohorts'',
represented by \(\lambda_i\) and \(\mu_i\), these being the total number
and mean size of individuals within the cohort respectively. Under this
assumption, the rates of change \(\frac{\rm d} {{\rm d}a} \lambda_i\)
and \(\frac{\rm d} {{\rm d}a} \mu_i\) can be approximated given by two
ODEs, which can in turn be approximated by first-order closed-form
solutions. Eq. \(\ref{eq:PDE}\) is thereby reduced to a family of ODEs,
which can be stepped using an appropriate ODE solver with an adaptive
step size. The size distribution \(n(x,h,a)\) is then approximated by a
series of point masses with position and amplitude given by
\(\lambda_i\) and \(\mu_i\).

In the current implementation we use a slightly different approach for
approximating the size distribution \(n(x,h,a)\). Instead of tracking
the first and second moments of the size-distribution within cohorts, we
track a point mass estimate of \(n\) situated on the characteristic
corresponding to cohort boundary. By integrating along characteristics,
the density of individuals with time of birth \(a_{0}\) is given by
\citep{Deroos-1997}:
\begin{equation}\label{eq:boundN}
  n(x, m, a)  =n(x,h_0 ,a_0) 
   \exp \left(-\int _{a_0}^{a} \left[\frac{\partial g(x,h(x, a_0, a^\prime),E_{a^\prime})}{\partial h} +d(x,h(x, a_0, a^\prime),E_{a^\prime})\right] {\rm d}a^\prime \right).
\end{equation}
Eq. \ref{eq:boundN} states that the density \(n\) at a specific time is
a product of density at the origin adjusted for changes through
development and mortality. Density decreases through time because of
mortality, as in a standard typical survival equation, but also changes
due to growth. If growth is slowing with size, (i.e.
\(\partial g / \partial h < 0\)), then density increases as the
characteristics compress, or density increases if
\(\partial g / \partial h > 0\).

If \(\left[h_0, h_+ \right)\) represents the entire state-space
attainable by any individual, the EBT algorithm proceeds by sub-dividing
this space into a series of cohorts with boundaries
\(h_0 < h_1 < \ldots < h_k\). These cohorts are then transported along
the characteristics of Eq. \(\ref{eq:PDE}\). The location of cohort
boundaries is controlled indirectly, via the schedule of time sat which
new cohorts are initialised into the population. We then track the
demography of a hypothetical individual located at each cohort boundary.

The equations below outline how to solve for the size, survival, seed
output and abundance of individuals located on the cohort boundaries.
Each of these problems is formulated as an Initial-Value ODE Problem
(IVP), which can then be solved using an ODE stepper.

\subsection{Size}\label{size}

The size of individual located on the boundary is obtained via Eq.
\ref{eq:size}, which is solved via the IVP:
\[\frac{dy}{dt} = g(x,y,t),\] \[ y(0) = h_0.\]

\subsection{Survival}\label{survival}

The probability of an individual located on the boundary surviving from
\(a_0 \rightarrow a\) is obtained via Eq. \ref{eq:survivalIndividual},
which is solved via the IVP:
\[\frac{dy}{dt} = d(x,h_i(a^\prime), E_{a^\prime}),\]
\[ y(0) = - \ln\left(S_{\rm G} (x^\prime,h_0, E_{a_0})\right) .\]
Survival is then \[ S_{\rm I} (x, a_0, a) = \exp\left(- y(a) \right).\]

\subsection{Seed production}\label{seed-production}

The lifetime seed production of individuals located on the boundary is
obtained via Eq. \ref{eq:tildeR}, which is solved via the IVP:
\[\frac{dy}{dt} = S_D \,f(x, m_i(a^\prime), E_{a^\prime}) \, S_{\rm I} (x, a_0, a^\prime) \, S_{\rm P} (a_0,a^\prime),\]
\[ y(0) = 0,\] where \(S_{\rm I}\) is individual survival (defined
above) and \(S_{\rm P}\) is calculated as in Eq.
\(\ref{eq:survivalPatch}\).

\subsection{Density of individuals}\label{density-of-individuals}

The density of individuals located on the boundary is obtained via Eq.
\ref{eq:boundN}, which is solved via the IVP:
\[\frac{dy}{dt} = \frac{\partial g(x,h_i(a^\prime), E_{a^\prime})}{\partial h} +d(x,h_i(a^\prime),E_{a^\prime}),\]
\[ y(0) = -\ln\left(n(x,h_0 ,a_0) \, S_{\rm G} (x^\prime,h_0, E_{a_0}) \right).\]
Density is then given by \[n(x,h_0 ,a_0) =\exp(-y(a)).\]

\section{Controlling error in the {\plant} model}
\label{controlling-error-in-the-plant-model}

In this section we outline how to control the error of the solution
estimated to the system described in the previous two sections.
Numerical solutions are required to solve a variety of problems. To
estimate the amount of light at a given height in a patch requires that
we integrate over the size-distribution within that patch. Calculating
assimilation for a plant in turn requires that we integrate
photosynthesis over this light profile. Approximating patch dynamics
requires that identify a vector of times where new cohorts are
introduced, then step the equations for each cohort forward in time to
estimate their size, survival and fecundity at different time points.
Root solving is required to solve the initial height of a plant given
it's seed mass, and the equilibrium seed rains across the
metapopulation. As with all numerical techniques, the solutions are
accurate only up to some specified level. These levels are controlled
via parameters found within the \texttt{control} object. Below we
provide a brief overview of the different numerical techniques being
applied and outline how error tolerance can be increased or decreased. 
We refer to various control parameters, which can be found within the 
\texttt{control} object. 

\subsection{Initial height of plants}\label{initial-height-of-plants}

When a seed germinates it produces a seedling of a given height. The
height of these seedlings is assumed to vary with the seed mass;
however, because there is no analytical solution relating height to seed
mass -- at least using the default physiological model -- we must solve
this height numerically. The calculation is performed by the function
\texttt{height\_seed} within the strategy description, using the Boost
library's 1-D \texttt{bisect} routine
\citep{Schaling-2014, Eddelbuettel-2015}. The accuracy of the solution
is controlled by the parameter \texttt{plant\_seed\_tol}.

\subsection{Approximation of size-density distribution via the EBT}
\label{approximation-of-size-density-distribution-via-the-ebt}

Errors in the EBT approximation to \(n\) can arise from two sources: i)
poor spacing of cohorts in the size dimension, and ii) when stepping
cohorts through time.

The first factor controlling the accuracy with which cohorts are stepped
through time is the accuracy of the ODE stepper being used. {\plant} uses
the embedded Runge-Kutta Cash-Karp 4-5 algorithm \citep{Cash-1990}, with
code ported directivity from the
\href{http://www.gnu.org/software/gsl/}{GNU Scientific Library}
\citep{Galassi-2009}. Accuracy of the solver is controlled by two
control parameters for relative and absolute accuracy:
\texttt{ode\_tol\_rel} and \texttt{ode\_tol\_abs}.

A second factor controlling the accuracy with which cohorts are stepped
through time is the accuracy of the derivative calculation in Eq.
\ref{eq:boundN}, calculated via standard finite differencing
\citep{Abramowitz-2012}. When the parameter
\texttt{cohort\_gradient\_richardson} is TRUE a Richardson Extrapolation
\citep{Stoer-2002} is used to refine the estimate, up to depth
\texttt{cohort\_gradient\_richardson}. The overall accuracy of the
derivative is controlled by \texttt{cohort\_gradient\_eps}.

The primary factor controlling the spacing of cohorts is the schedule of
cohort introduction times -- a vector of times indicating times at which
a new cohort is initialised. Because the system is deterministic, the
schedule of cohort introduction times determines the spacing of cohorts
through the entire development of the patch. Poor cohort spacing
introduces error because various emergent properties -- such as total
leaf area, biomass or seed output -- are estimated by integrating over
the size distribution. The accuracy of these integrations declines
directly with the spacing of cohorts. Thus we aim to build an
appropriately refined schedule, which allows required integrations to be
performed within the desired accuracy at every time point. At the same
time, we want as few cohorts as possible to maintain for computational
efficiency.

A suitable schedule is found using the function
\texttt{build\_schedule}. As there was no prior existing method for
estimating a suitable schedule of introduction times, we implemented the
following new technique. The \texttt{build\_schedule} function takes an
initial vector of introduction times, then looks at each cohort and asks
whether removing that cohort would cause the error introduced when
integrating two specified functions over the size distribution to jump
over the allowable error threshold \texttt{schedule\_eps}. This
calculation is repeated for every time step in the development of the
patch. A new cohort is introduced immediately prior to any cohort
failing the above tests. The dynamics of the patch are
then simulated again and the process repeated, until all integrations at
all time points have error below the tolerable limit
\texttt{schedule\_eps}. Decreasing \texttt{schedule\_eps} demands higher
accuracy from the solver and thus increases the number of cohorts being
introduced. Note we are asking whether removing an existing cohort would
cause error to jump above the threshold limit, and using this to decide
whether an extra cohort -- in addition to the one used in the test --
should be introduced. Thus the actual error is likely to
be lower than but at most equal to \texttt{schedule\_eps}.

To determine the error associated with a specified cohort, we integrate
two different functions over the size distribution, within the function
\texttt{run\_ebt\_error}. We then asses how much removing the focal
cohort would increase the error in the two integrations. The first
integration, performed by the function \texttt{area\_leaf\_error}, asks
how much removal of the focal cohort would increase error in the
estimate of total leaf area in the patch. The second integration,
performed by the function \texttt{seed\_rain\_error}, asks how much
removal of the focal cohort would increase error in the total seed
produced from the patch. The relative error in each integration is then
calculated using the \texttt{local\_error\_integration} function.

\subsection{Calculation of light environment and influence on
assimilation}\label{calculation-of-light-environment-and-influence-on-assimilation}

To progress the system of ODE's requires that we calculate the amount of
shading on each of the cohort boundaries, from all other plants in the
patch.

Calculating canopy openness at a given height \(E_a(z)\) requires that
we integrate over the size distribution (Eq. \ref{eq:light}). This
integration is performed using the trapezium rule, within the function
\texttt{area\_leaf\_above} in \texttt{species.h}. The main factor
controlling the accuracy of the integration is the spacing of cohorts.
As outlined in the section above, cohort introduction times control the
spacing of cohorts and are refined so that the integration of leaf area
is within some specified limit. Thus the simple trapezium integration
within the \texttt{area\_leaf\_above} function \emph{is} refined
adaptively via the \texttt{build\_schedule} function.

The cost of calculating \(E_a(z)\) increases linearly with the number of
cohorts in the system. But the same calculation must then be repeated
many times (multiple heights for every cohort), so the overall CPU cost
of stepping the system increases as to \(O(k^2)\), where \(k\) is the
total number of cohorts across all species. This disproportionate
increase in CPU cost with the number of cohorts is highly undesirable.

We reduced the computational cost of the continuous competitive feedback
from \(O(k^2) \rightarrow O(k)\), by approximating the \(E_a(z)\) with a
spline. Biologically, Eq. \ref{eq:light} is a simple monotonically
increasing function of size. This function is easily
approximated using a piece-wise continuous spline fitted to a limited
number of points. Once fitted, the spline can be used to estimate any
additional evaluations of competitive effect, and since spline
evaluations are cheaper than integrating over the size distribution,
this approach reduces the overall cost of stepping the resident
population. A new spline is then constructed at the next time step.

The accuracy of the spline interpolation depends on the number of points
used in its construction and their location along the size axis. We
select the location and number of points via an adaptive algorithm.
Starting with an initial set of 33 points, we assess how much each point
contributes to the accuracy of the spline fit at the location of each
cohort first via exact calculation, and second by linearly interpolating
from adjacent cohorts. The absolute difference in these values is
compared to the control parameter \texttt{environment\_light\_tol}. If
the error value is greater than this value, the interval is bisected and
process repeated. For full details see
\texttt{adaptive\_interpolator.h}.

\subsection{Integrating over light
environment}\label{integrating-over-light-environment}

Plants have leaf area distributed over a range of heights; estimating
assimilation of a plant at each time step thus requires us to integrate
leaf-level rates over the plant. The integration is performed using
Gaussian quadrature, using the QUAD PACK routines \citep{Piessens-1983},
adapted from the \href{http://www.gnu.org/software/gsl/}{GNU Scientific
Library}\citep{Galassi-2009} (see \texttt{qag.h} for further details).
If the control parameter \texttt{plant\_assimilation\_adaptive} is TRUE,
the integration is performed using adaptive refinement with accuracy
controlled by the parameter \texttt{plant\_assimilation\_tol}.

\subsection{Solving demographic seed
rain}\label{solving-demographic-seed-rain}

For a single species, solving for \(y_x\) is a straightforward
one-dimensional root-finding problem, which can be solved with accuracy
\texttt{equilibrium\_eps} via a simple bisection algorithm (see 
\texttt{equlibrium.R} for details). 

Solving seed rains in multi-species systems is significantly harder,
because there is no sure method for multi-dimensional root finding. We
have implemented several different approaches (see 
\texttt{equlibrium.R} for details).


\clearpage

\section{Tables}\label{tables}

\begin{table}[ht]
 \caption{Variable names and definitions in demographic model.}
\centering
  \begin{tabular}{p{2cm}p{2cm}p{9cm}}
  \hline
  Symbol & Unit & Description \\
  \hline
  \multicolumn{3}{l}{\textbf{Patch state variables}} \\
  $N$   & & number of species \\
  $a, a^{\prime}$ & yr & patch age \\
  $a_0$ & yr & patch age when plant germinates \\
  $E_a$ & & profile of canopy opennesses within patch age $a$\\
  $E_a(z)$& 0-1 & canopy opennesses at height $z$ within a patch age $a$\\

  \multicolumn{3}{l}{\textbf{Plant state variables}} \\
  $x$   & & vector of traits for a species\\
  $h$   & m & height of plant\\
  $h_0$   & m  & height of seedling after germinating\\
  $z$   & m & height in canopy\\
  $a_l(h)$  & m$^{-2}$ & leaf area of plant height $h$ \\
  $Q(z, h)$ & 0-1 & fraction of leaf area for plant height $h$ above $z$\\
 
  \multicolumn{3}{l}{\textbf{Abundance measures}} \\
  $p(a)$ & yr$^{-1}$ & frequency-density of patches age $a$ \\
  $y_x$ & m$^{-2}$ yr$^{-1}$ & vector of seed rains for a species with traits $x$\\
  $n(x,h,a)$ & m$^{-1}$ m$^{-2}$ & density of plants per unit height per ground area\\

  \multicolumn{3}{l}{\textbf{Demographic rates}} \\
  $g(x,h, E_a)$ & m yr$^{-1}$ & height growth rate \\
  $d(x,h, E_a)$ & yr$^{-1}$ & instantaneous mortality rate \\
  $f(x,h, E_a)$ & yr$^{-1}$ & seed production rate \\
  $\gamma(a)$ & yr$^{-1}$ & instantaneous disturbance rate\\

  \multicolumn{3}{l}{\textbf{Demographic outcomes}} \\
  $h(x,a_0,a)$   & m  & height of plant germinating in patch age $a_0$ at age $a$\\
  $S_{\rm D}$ & 0-1 & probability a seed survives dispersal \\
  $S_{\rm G} (x,h_0, E_{a_0})$ & 0-1 & probability a seed germinates successfully \\
  $S_{\rm I} (x, a_0, a)$ & 0-1 & probability an individual survives from $a_0$ to $a$\\
  $S_{\rm P} ( a_0, a)$ & 0-1 & probability a patch remains undisturbed from $a_0$ to $a$\\
  $\tilde{R}(x, a_0, a)$ & & cumulative seed output for plant from $a_0$ to $a$ \\
  $R\left(x^\prime,x\right)$ & & basic reproduction ratio for mutant growing in environment of residents \\
 
  \multicolumn{3}{l}{\textbf{Miscellaneous constants}} \\
  $c_\textrm{ext}$ & 0-1  & light extinction coefficient\\
  \hline
  \end{tabular}
\label{tab:definitions}
\end{table}

\clearpage

\section{Appendices}\label{appendices}

\subsection{Derivation of PDE describing age-structured
dynamics}\label{derivation-of-pde-describing-age-structured-dynamics}

Consider patches of habitat which are subject to some intermittent
disturbance and where the age of a patch is corresponds to the time
since the last disturbance. Let \(p(a,t)\) be the frequency-density of
patches age \(a\) at time \(t\) and let \(\gamma(a)\) be the
age-dependent probability that a patch of age \(a\) is transformed into
a patch of age \(0\) through disturbance. Then according to the Von
Foerster equation for age-structured population dynamics
\citep{Vonfoerster-1959}, the dynamics of \(p(a,t)\) are given by
\[ \frac{\partial }{\partial t} p(a,t)=-\frac{\partial }{\partial a} p(a,t)-\gamma(a,t)p(a,t),\]
with boundary condition
\[ p(0,t)=\int^{\infty}_{0}\gamma(a,t)p(a,t)\,{\rm d}a.\]

The frequency of patches with \(a < x\) is given by
\(\int_{0}^{x}p(a,t) \, {\rm d}a\), with
\(\int_{0}^{\infty} p(a,t) \, {\rm d}a =1\). If
\(\frac{\partial}{\partial t}\gamma(a,t)=0\), then \(p(a)\) will
approach an equilibrium solution given by \[p(a) = p(0) S_{\rm P}(0,a),\] where
\[S_{\rm P}(0,a) = \exp \left( \int_{0}^{a}-\gamma(\tau)\,{\rm d}\tau\right)\]
is the probability that a local population will remain undisturbed for
at least \(a\) years (patch survival function), and
\[p(0) = \frac1{ \int_{0}^{\infty}S_{\rm P}(0,a) \,{\rm d}a}\] is the
frequency-density of patches age 0. The rate of disturbance for patches
age \(a\) is given by
\(\frac{\partial (1-S_{\rm P}(0,a))}{\partial a} = \frac{-\partial S_{\rm P}(0,a)}{\partial a}\),
while the expected lifetime for patches is
\(\int_0^\infty - a \frac{\partial}{\partial a} S_{\rm P}(0,a) \, {\rm d} a = \int_0^\infty S_{\rm P}(0,a) \, {\rm d} a = \frac1{p(0)}\)
(2\(^{nd}\) step made using integration by parts).

An equilibrium patch age distribution may be achieved under a variety of
conditions, for example if \(\gamma(a,t)\) depends on patch age but the
this probability is constant over time. The probability of disturbance
may also depend on features of the vegetation (rather than age \emph{per
se}), in which case an equilibrium is still possible, provided the
vegetation is also assumed to be at equilibrium.

\subsubsection{Exponential distribution}\label{exponential-distribution}

If the probability of patch disturbance is constant with respect to
patch age (\(=\lambda\)), then rates at which patches age \(a\) are
disturbed follow an exponential distribution:
\(-\partial S_{\rm P}(0,a)/ \partial a = \lambda e^{-\lambda a}\). The patch age
distribution is then given by:
\[ S_{\rm P}(0,a) = \exp\left(-\lambda a\right), p(0) = \lambda.\]

\subsubsection{Weibull distribution}\label{weibull-distribution}

If the probability of patch disturbance changes as a function of time,
with \(\gamma(a) = \lambda \psi a^{\psi-1}\), then rates at which
patches age \(a\) are disturbed follow a Weibull distribution:
\(-\partial S_{\rm P}(0,a)/ \partial a = \lambda \psi a^{\psi -1}e^{-\lambda a^\psi}\).
Values of \(\psi>1\) imply probability of disturbance increases with
patch age; \(\psi<1\) implies probability of disturbance decreases with
age. When \(\psi=1\) we obtain the exponential distribution, a special
case of the Weibull. The Weibull distribution results in following for
the patch age distribution:
\[S_{\rm P}(0,a) = e^{-\lambda a^\psi}, p(0) =  \frac{\psi \lambda^{\frac1{\psi}}}{\Gamma\left(\frac1{\psi}\right)},\]
where \(\Gamma(x)\) is the gamma function
\(\left(\Gamma(x) = \int_{0}^{\infty}e^{-t}t^{x-1}\, dt\right)\). We can
also specify the distribution by its mean return time
\(\bar{a} = \frac1{p(0)}\). Then, calculate relevant value for
\(\lambda = \left(\frac{\Gamma\left(\frac1{\psi}\right)}{\psi \bar{a}}\right)^{\psi}\).

\subsubsection{Variable distributions}\label{variable-distributions}

The probability of patch disturbance might also be vary with properties
of the vegetation. In this case, we cannot prescribe a known
distribution to \(p(a)\), it must be solved numerically. Patch survival
can be calculated numerically as
\[S_{\rm P}(0,a) = \exp \left( \int_{0}^{a}-\gamma(\tau)\,{\rm d}\tau\right).\]
For calculations of fitness, we want to integrate over some fraction
\(q\) of the patch age distribution (i.e.~ignoring the long tail of the
distributions). Thus we want to find the point \(x\) which satisfies
\(\int_{0}^{x} p(a)\,{\rm d}a =q\).Locating \(x\) requires knowledge of
\(p(0)\), which in turn requires us to approximate the tail of the
integral \(\int_{0}^{\infty} S_{\rm P}(0,a)\,{\rm d}a\). For \(a > x\), let
\(S_{\rm P}(0,a)\) be approximated by
\[S_{\rm P}(0,a) \approx S_{\rm P}(0,x) \exp\left( - (x-a) \gamma(x)\right).\] Then
\[p(0)^{-1} =\int_{0}^{\infty} S_{\rm P}(0,a)\,{\rm d}a \approx \int_{0}^{x} S_{\rm P}(0,a)\,{\rm d}a + \int_{x}^{\infty } S_{\rm P}(0,a)\,{\rm d}a\]
\[ = \int_{0}^{x} S_{\rm P}(0,a)\,{\rm d}a + \frac{S_{\rm P}(0,x)}{\gamma(x)}. \]
Substituting into \(\int_{0}^{x} p(a)\,{\rm d}a =q\), we obtain
\[q=  p(0) \int_{0}^{x} S_{\rm P}(0,a) \, {\rm d}a = \frac{\int_{0}^{x} S_{\rm P}(0,a) \, {\rm d}a}{\int_{0}^{x} S_{\rm P}(0,a)\,{\rm d}a + \frac{S_{\rm P}(0,x)}{\gamma(x)}}\]
\[\Rightarrow  \frac{S_{\rm P}(0,x)}{\gamma(x)} = \frac{1-q}{q} \int_{0}^{x} S_{\rm P}(0,a) \, {\rm d}a.\]
Thus by monitoring
\(S_{\rm P}(0,x), \gamma(x), \int_{0}^{x} S_{\rm P}(0,a) \, {\rm d}a\) we can evaluate
when a sufficient range of the patch age distribution has been
incorporated.

\subsection{Derivation of PDE describing size-structured
dynamics}\label{derivation-of-pde-describing-size-structured-dynamics}

To model the population we use a PDE describing the dynamics for a thin
slice \(\Delta h\). Assuming that all rates are constant within the
interval \(\Delta h\), the total number of individuals within the
interval spanned by \([h-0.5\Delta h,h+0.5\Delta h)\) is
\(n(h,a)\Delta h\) . The flux of individuals in and out of the size
class can be expressed as
\begin{equation}\begin{array}{ll} J(h,a)=&g(h-0.5 \Delta h,a) \, n(h-0.5 \Delta h,a)-g(h+0.5 \Delta h,a) \, n(h+0.5 \Delta h,a) \\ &-d (h,a) \, n(h,a)\Delta h\\ \end{array}.
\end{equation}
The first two terms describe the flux in and out of the size class
through growth; the last term describes losses due through mortality.
The change in number of individuals within the interval across a time
step \textit{$\Delta $t} is thus:
\begin{equation}
  \begin{array}{ll} n(h,a+\Delta a)\Delta h-n(h,a)\Delta h= &g(h-0.5 \Delta h,a) \, n(h-0.5 \Delta h,a)\Delta a \\ &-g(h+0.5 \Delta h,a) \, n(h+0.5 \Delta h,a)\Delta a\\&-d (h,a) \, n(h,a)\Delta h\Delta a.
  \end{array}
\end{equation}
Rearranging,
\begin{equation}
  \begin{array}{ll}
  \frac{n(h,a+\Delta a)-n(h,a)}{\Delta a} = &-d (h,a) \, n(h,a) \\
  &-\frac{g(h+0.5 \Delta h,a) \, n(h+0.5 \Delta h,a)-g(h-0.5 \Delta h,a) \, n(h-0.5 \Delta h,a)}{\Delta h}.
  \end{array}
\end{equation}
The LHS is corresponds to the derivative of \(n\) as \(\Delta a\to 0\).
For thin slices, \(\Delta h \sim 0\), we obtain
\begin{equation} \label{eq:PDE-app}
  \frac{\partial }{\partial t} n(h,a)=-d (h,a) \, n(h,a)-\frac{\partial }{\partial h} (g(h,a) \, n(h,a)).
\end{equation}

To complete the model, the PDE must be supplemented with boundary
conditions that specify the density at the lower end of \(n(h_{0},a)\)
for all \(t\) as well as the the initial distribution when \(t=0\),
\(n(h,0)\). The former is derived by integrating the PDE with respect to
\(h\) over the interval \((h_{0},h_{\infty })\), yielding
\begin{equation}\frac{\partial }{\partial t} \int _{h_{0} }^{h_{\infty } }n(h,a) \partial h=g(h_{0} ,a) \, n(h_{0} ,a)-g(h_{\infty } ,a) \, n(h_{\infty } ,a)-\int _{h_{0} }^{h_{\infty } }d (h,a) \, n(h,a) \partial h.
\end{equation}

The LHS of this relationship is evidently the rate of change of total
numbers of individual in the population, while the right-hand-term is
the total population death rate. Further, \(n(h_{\infty } ,a)=0\). Thus
to balance total births and deaths, \(g(h_{0} ,a) \, n(h_{0} ,a)\) must
equal the birth rate \(B(x, a)\). Thus the boundary condition is given
by
\begin{equation}g(h_{0} ,a) \, n(h_{0} ,a)=B(x, a).
\end{equation}

\subsection{Converting density from one size unit to
another}\label{converting-density-from-one-size-unit-to-another}

Population density is explicitly modelled in relation to a given size
unit (Eq. \(\ref{eq:PDE}\)). But what if we want to express density in
relation to another size unit? A relation between the two can be derived
by noting that the total number of individuals within a given size range
must be equal. So let's say density is expressed in units of size \(m\),
but we want density in units of size \(h\). First we require a
one-to-one function which h for a given \(m\): \(h = \hat{h}(m)\). Then
the following must hold
\begin{equation} \label{eq:n_conversion} \int_{m_1}^{m_2} n(x,m,a) \, \textrm{d}m =  \int_{\hat{h}(m_1)}^{\hat{h}(m_2)} n^\prime(x,m,a) \, \textrm{d}h
\end{equation}
For very small size intervals, this equation is equivalent to
\begin{equation} \left(m_2- m_1 \right) \, n(x,m_1,a) = \left( \hat{h}(m_2) - \hat{h}(m_1)\right) \, n^\prime(x, \hat{h}(m_1),a).
\end{equation}
Rearranging gives
\begin{equation}  n^\prime(x, \hat{h}(m_1),a) = n(x, m_1,a) \, \frac{m_2- m_1}{\hat{h}(m_2) - \hat{h}(m_1)}
\end{equation}
Noting that the second term on the RHS is simply the definition of
\(\frac{\delta m}{\delta h}\) evaluated at \(m_1\), we have
\begin{equation} \label{eq:n_conversion2} n^\prime(x, h, a) = n(x, m,a) \, \frac{\delta m}{\delta h}.
\end{equation}

\clearpage
\bibliography{../refs}

\end{document}
